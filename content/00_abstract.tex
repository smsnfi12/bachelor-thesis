% \thispagestyle{plain}

% \section*{Kurzfassung}
% Hier steht eine Kurzfassung der Arbeit in deutscher Sprache inklusive der Zusammenfassung der
% Ergebnisse.
% Zusammen mit der englischen Zusammenfassung muss sie auf diese Seite passen.

% \section*{Abstract}
% \begin{foreignlanguage}{english}
% The abstract is a short summary of the thesis in English, together with the German summary it has to fit on this page.
% \end{foreignlanguage}
\section*{Abstract}
Das IceCube-Neutrino-Observatorium ist ein Detektor mit einem Volumen von etwa einem Kubikkilometer, der sich innerhalb des Eises am geografischen Südpol befindet. 
Der Detektor ist mit einem dreidimensionalen Gitter von über 5000 Photomultiplier-Einheiten ausgestattet, die es IceCube durch die präzise Messung von 
Cherenkov-Strahlung ermöglichen, Signale astrophysikalischer Quellen, insbesondere von Neutrinos zu untersuchen. Diese Arbeit fokussiert sich auf die Untersuchung
koinzidenter Myon-Events. Dabei liegt ein weiterer Fokus auf Daten des Fixed Rate Triggers (FRT), welcher einen ungefilterten Einblick in die gemessenen Signale ermöglicht.
Hierbei wird ein theoretisches Modell für die Berechnung der Häufigkeit koinzidenter Events unter dem Einfluss verschiedener Variabeln aufgestellt. 
Darüber hinaus 
werden die Daten des FRT detailliert analysiert, mit dem Versuch, Eigenschaften zu klassifizieren, die Hinweise auf koinzidente Events darstellen könnten. \\ \\

The IceCube Neutrino Observatory is a detector with a volume of approximately one cubic kilometer, located within the ice at the geographic South Pole. The detector 
is equipped with a three-dimensional grid of over 5,000 photomultiplier units, which enable IceCube to investigate signals from astrophysical sources, particularly 
neutrinos, through the precise measurement of Cherenkov radiation. This study focuses on the investigation of coincident muon events, with additional emphasis on 
data from the Fixed Rate Trigger (FRT), which provides an unfiltered view of the recorded signals. A theoretical model was developed to calculate the frequency of 
coincident events under the influence of various variables. Furthermore, the FRT data is analyzed in detail, aiming to classify features that may indicate coincident 
events.

