\chapter{Discussion of Results}\label{chap:discussion}

\section{Theoretical Calculations}
The calculations of the theoretical coincident muon event rate in section~\ref{sec:muon_coincidence} provide a simple framework for estimating 
coincident event probabilities under varying trigger windows. As seen in figure~\ref{fig:coin_rate_combined}, the theoretical coincidence probability 
exceeds \SI{1}{\percent} for a readout window of \SI{5}{\micro\second} for significant intervals of both the energy- and the zenith angle spectrum. 
Even if this probability is subject to further variables, as it most likely is, it appears that the magnitude of coincident events is at least 
significant enough to consider for precise analyses. As the analysis here is only based on Monte Carlo simulations, similar evaluations based on 
experimental data could be interesting. Also, further variables could be taken into account.
Based on where in the detector two or more events are detected around the same time, the spatial and geometric conditions for certain triggers could have 
an impact on the likelihood of a coincident getting detected as such. 

\section{Fixed Rate Trigger}
Analyzing the FRT data in detail serves as an interesting insight into the unfiltered patterns of detected signals in IceCube. The comparison of filtered 
and unfiltered signals revealed, that filtering by physically relevant signals does not justify any cutoff at low charges, as shown in figure~\ref{fig:frt_mu_sub_comp_1}.
While this reality created significant challenges for the subsequent analyses, it is a valuable finding and should be analyzed further. 
The main challenge arising from the lack of a clear distinction between signals and possible noise was, that characterizing possible event or coincident event candidates 
proved difficult. As a result, a number of the analyses created for the FRT data were left with no concrete results. 
However, some interesting features were visible when analyzing the charge over time functions for different subevents, as shown in figure~\ref{fig:high_low_comp}.
This led to the attempt of characterizing possible coincident events. This attempt did not yield the desired level of success. As the features of coincident events 
are highly complex, the relatively simple approach of finding peaks in a projection of charge and frequency could be considered an oversimplification.
Still, inspecting the charges and counts of signals in the subevents as a function of time might be a valuable approach to finding coincident events, when combined
with different methods to search for clusters. Machine learning algorithms could be used for more a more complex search for distinguishing characteristics.

\section{Conclusion}
While the unsatisfactory lack of concrete results is undeniable, valuable insights were found on the FRT, which would not be accessible by analyzing any other 
datasets available for IceCube. Further evaluation and the use of machine learning algorithms could possibly lead to more concrete results in the future.