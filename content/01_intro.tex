\chapter{Introduction}

The IceCube Neutrino Observatory, located at the geographic South Pole, is the largest particle detector of its kind, designed to detect high-energy neutrinos originating 
from astrophysical sources. One of the challenges in IceCube's data analysis lies in identifying coincident muon events, which occur when two or more muons or muon bundles
from separate 
primary particles reach the detector within a short time window. Understanding these events provides valuable insights into cosmic ray interactions and the 
detector's capabilities.
This thesis focuses on analyzing coincident events in IceCube with a particular emphasis on the Fixed Rate Trigger (FRT), a unique data acquisition method that 
records unfiltered signals across the entire detector. Unlike other triggers that apply stringent filtering criteria, the FRT provides a comprehensive view of 
unfiltered signals, making it ideal for studying coincident events. However, the lack of easily measurable differences between signals and noise poses challenges 
in distinguishing significant signals from background noise.
To address this, a combination of theoretical modeling and data analysis was employed. Theoretical calculations based on Poisson statistics were used to estimate 
coincidence rates, while a Python-based peak detection algorithm was applied to identify significant subevents within FRT data. 
This thesis is structured as follows: Chapter~\ref{chap:theory} provides the theoretical background, including an overview of the IceCube detector, the physics of Cherenkov 
radiation, and the concept of coincident events. Chapter~\ref{chap:results} presents the results of the analysis, including the theoretical estimation of coincidence probabilities 
and the analysis of FRT data. The concluding discussion highlights the broader implications of the findings and potential directions for future research.
